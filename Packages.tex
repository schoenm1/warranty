% Anpassung des Seitenlayouts ----------------------------------------------
% 	siehe Seitenstil.tex
% --------------------------------------------------------------------------
\usepackage[
	automark,			% Kapitelangaben in Kopfzeile automatisch erstellen
	headsepline,	% Trennlinie unter Kopfzeile
	ilines				% Trennlinie linksb�ndig ausrichten
]{scrpage2}


% Anpassung an Landessprache -----------------------------------------------
% 	Verwendet globale Option german siehe \documentclass
% --------------------------------------------------------------------------
\usepackage{babel}


% Umlaute ------------------------------------------------------------------
% 	Umlaute/Sonderzeichen wie ���� direkt im Quelltext verwenden (CodePage).
%		Erlaubt automatische Trennung von Worten mit Umlauten.
% --------------------------------------------------------------------------
\usepackage[latin1]{inputenc}
\usepackage[T1]{fontenc}
%\usepackage{ae} % "sch�neres" �
\usepackage{textcomp} % Euro-Zeichen etc.

% Grafiken -----------------------------------------------------------------
% 	Einbinden von EPS-Grafiken [draft oder final]
% 	Option [draft] bindet Bilder nicht ein - auch globale Option
% --------------------------------------------------------------------------
\usepackage[dvips,final]{graphicx}
\graphicspath{{Bilder/}} % Dort liegen die Bilder des Dokuments

% Befehle aus AMSTeX f�r mathematische Symbole z.B. \boldsymbol \mathbb ----
\usepackage{amsmath,amsfonts}


% Weitere Zeichen z.B. \textcelsius \textordmasculine \textsurd \textonehalf 
% \texteuro \texttimes \textdiv ... aus textcomp.sty
% siehe >>Schnell ans Ziel mit \LaTeXe<< von J�rg Knappen
% (Oldenbourg, M�nchen und Wien 1997, ISBN 3-486-24199-0)
% \usepackage{tccompat}
	

% F�r Index-Ausgabe; \printindex -------------------------------------------
\usepackage{makeidx}


% Einfache Definition der Zeilenabst�nde und Seitenr�nder etc. -------------
\usepackage{setspace}
\usepackage{geometry}


% Symbolverzeichnis --------------------------------------------------------
% 	Symbolverzeichnisse bequem erstellen, beruht auf MakeIndex.
% 		makeindex.exe %Name%.nlo -s nomencl.ist -o %Name%.nls
% 	erzeugt dann das Verzeichnis. Dieser Befehl kann z.B. im TeXnicCenter
%		als Postprozessor eingetragen werden, damit er nicht st�ndig manuell
%		ausgef�hrt werden muss.
%		Die Definitionen sind ausgegliedert in die Datei Abkuerzungen.tex.
% --------------------------------------------------------------------------
\usepackage[intoc]{nomencl}
  \let\abbrev\nomenclature
  \renewcommand{\nomname}{Abk�rzungsverzeichnis}
  \setlength{\nomlabelwidth}{.25\hsize}
  \renewcommand{\nomlabel}[1]{#1 \dotfill}
  \setlength{\nomitemsep}{-\parsep}


% Zum Umflie�en von Bildern -------------------------------------------------
\usepackage{floatflt}


% Zum Einbinden von Programmcode --------------------------------------------
\usepackage{listings}
\usepackage{xcolor} 
\definecolor{hellgelb}{rgb}{1,1,0.9}
\definecolor{colKeys}{rgb}{0,0,1}
\definecolor{colIdentifier}{rgb}{0,0,0}
\definecolor{colComments}{rgb}{1,0,0}
\definecolor{colString}{rgb}{0,0.5,0}
\lstset{%
    float=hbp,%
    basicstyle=\texttt\small, %
    identifierstyle=\color{colIdentifier}, %
    keywordstyle=\color{colKeys}, %
    stringstyle=\color{colString}, %
    commentstyle=\color{colComments}, %
    columns=flexible, %
    tabsize=2, %
    frame=single, %
    extendedchars=true, %
    showspaces=false, %
    showstringspaces=false, %
    numbers=left, %
    numberstyle=\tiny, %
    breaklines=true, %
    backgroundcolor=\color{hellgelb}, %
    breakautoindent=true, %
%    captionpos=b%
}

% Lange URLs umbrechen etc. -------------------------------------------------
\usepackage{url}


% Wichtig f�r korrekte Zitierweise ------------------------------------------
\usepackage[square]{natbib}
% Quellenangaben in eckige Klammern fassen ---------------------------------
\bibpunct{[}{]}{;}{a}{}{,~}


%\usepackage{jurabib}
%\jurabibsetup{authorformat=smallcaps,% Autor in Kapit�lchen              
%              %authorformat=year,
%              authorformat=citationreversed,% Im Zitat Vorname vorne
%              authorformat=indexed,% Autor in Index
%              authorformat=and,% Autoren mit "," und "und" abgetrennt
%              authorformat=firstnotreversed,%
%              authorformat=reducedifibidem,% Bei Verweis nur Nachname. 
%              %superscriptedition=all,% Auflage hochgestellt
%              %citefull=first,% Erstzitat voll
%              titleformat=italic,              
%              titleformat=all,
%              titleformat=colonsep,% Doppelpunkt zwischen Aut. u. Titel
%              ibidem=strict,% Ebenda pro Doppelseite
%              see,% Das zweite Argument ist optional f�r "Vgl." etc.
%              commabeforerest,% Komma vor Seitenzahl
%              %howcited=compare,%
%              %bibformat=ibidem,% Strich bei widerholtem Autor in BIB.
%              commabeforerest,
%              bibformat=compress,
%              pages=always,
%              %pages=format,% S. wird vorweggesetzt
%              crossref=long,% Querverweise in voller L�nge
%              square,% eckige Klammern bei Zitaten
%              %oxford,
%              %chicago,
%}
%
%\AddTo\bibsgerman{% 
%\jblookforgender%
%\renewcommand*{\ibidemname}{Ebenda}%
%\renewcommand*{\ibidemmidname}{ebenda}% 
%\renewcommand*{\bibjtsep}{In: }% Vor Zeitschriften 
%\renewcommand*{\bibbtsep}{In: }% Vor Buchtitel
%\renewcommand*{\incollinname}{In: }%Nicht so ganz sauber. 
%\renewcommand*{\bibatsep}{.}% Nach Titel
%\renewcommand*{\bibbdsep}{}%Vor Datum 
%\renewcommand*{\jbaensep}{.}%
%\renewcommand*{\bibprdelim}{)}% Klammer bei Zeitschriftjahr rechts
%\renewcommand*{\bibpldelim}{(}% Klammer bei Zeitschriftjahr links
%\renewcommand*{\biblnfont}{\textsc}% Nachamen Autor im BIB
%\renewcommand*{\bibelnfont}{\textsc}% Nachamen Hg. im BIB
%\renewcommand*{\bibfnfont}{\textsc}% Vorn. Autor im BIB
%\renewcommand*{\bibefnfont}{\textsc}% Vorn. Hg. im BIB
%\renewcommand*{\jbcitationyearformat}[1]{#1}% Komma zwischen Autor und Jahr entfernen
%\def\herename{hier: }%
%\jbfirstcitepageranges% Format: S. x--z, hier y.  
%\renewcommand\bibidemSfname{\raisebox{.2em}{\rule{2.em}{.2pt}}~}%
%\renewcommand\bibidemsfname{\raisebox{.2em}{\rule{2.em}{.2pt}}~}%
%\renewcommand\bibidemPfname{\raisebox{.2em}{\rule{2.em}{.2pt}}~}%
%\renewcommand\bibidempfname{\raisebox{.2em}{\rule{2.em}{.2pt}}~}%
%\renewcommand\bibidemSmname{\raisebox{.2em}{\rule{2.em}{.2pt}}~}%
%\renewcommand\bibidemsmname{\raisebox{.2em}{\rule{2.em}{.2pt}}~}%
%\renewcommand\bibidemPmname{\raisebox{.2em}{\rule{2.em}{.2pt}}~}%
%\renewcommand\bibidempmname{\raisebox{.2em}{\rule{2.em}{.2pt}}~}%
%\renewcommand\idemSfname{Dies.}%
%\renewcommand\idemsfname{dies.}%
%\renewcommand\idemPfname{Dies.}%
%\renewcommand\idempfname{dies.}%
%\renewcommand\idemSmname{Ders.}%
%\renewcommand\idemsmname{ders.}%
%\renewcommand\idemPmname{Dies.}%
%\renewcommand\idempmname{dies.}%
%\renewcommand{\jbannoteformat}[1]{{\footnotesize\begin{quote}#1\end{quote}}}
%}%


%% PDF-Optionen -------------------------------------------------------------
\usepackage[
bookmarks,
bookmarksopen=true,
pdftitle={\titel},
pdfauthor={\autor},
pdfcreator={\autor},
pdfsubject={\titel},
pdfkeywords={\titel},
colorlinks=true,
linkcolor=red, % einfache interne Verkn�pfungen
anchorcolor=black,% Ankertext
citecolor=blue, % Verweise auf Literaturverzeichniseintr�ge im Text
filecolor=magenta, % Verkn�pfungen, die lokale Dateien �ffnen
menucolor=red, % Acrobat-Men�punkte
urlcolor=cyan, 
%linkcolor=black, % einfache interne Verkn�pfungen
%anchorcolor=black,% Ankertext
%citecolor=black, % Verweise auf Literaturverzeichniseintr�ge im Text
%filecolor=black, % Verkn�pfungen, die lokale Dateien �ffnen
%menucolor=black, % Acrobat-Men�punkte
%urlcolor=black, 
backref,
%pagebackref,
plainpages=false,% zur korrekten Erstellung der Bookmarks
pdfpagelabels,% zur korrekten Erstellung der Bookmarks
hypertexnames=false,% zur korrekten Erstellung der Bookmarks
linktocpage % Seitenzahlen anstatt Text im Inhaltsverzeichnis verlinken
]{hyperref}

% Zum fortlaufenden Durchnummerieren der Fu�noten ---------------------------
\usepackage{chngcntr}


% Aliase f�r Zitate
% \defcitealias{WPProzess}{Wikipedia:~Prozess}

%\usepackage{minitoc}

% f�r lange Tabellen
\usepackage{longtable}
\usepackage{array}
\usepackage{ragged2e}
\usepackage{lscape}

%Spaltendefinition rechtsb�ndig mit definierter Breite
\newcolumntype{w}[1]{>{\raggedleft\hspace{0pt}}p{#1}}

% Formatierung von Listen �ndern
\usepackage{paralist}
% \setdefaultleftmargin{2.5em}{2.2em}{1.87em}{1.7em}{1em}{1em}

% Anhangsverzeichnis
%\makeatletter% --> De-TeX-FAQ
%\newcommand*{\maintoc}{% Hauptinhaltsverzeichnis
%\begingroup
%\@fileswfalse% kein neues Verzeichnis �ffnen
%\renewcommand*{\appendixattoc}{% Trennanweisung im Inhaltsverzeichnis
%\value{tocdepth}=-10000 % lokal tocdepth auf sehr kleinen Wert setzen
%}%
%\tableofcontents% Verzeichnis ausgeben
%\endgroup
%}
%\newcommand*{\appendixtoc}{% Anhangsinhaltsverzeichnis
%\begingroup
%\edef\@alltocdepth{\the\value{tocdepth}}% tocdepth merken
%\setcounter{tocdepth}{-10000}% Keine Verzeichniseintr�ge
%\renewcommand*{\contentsname}{% Verzeichnisname �ndern
%Verzeichnis der Anh\"ange}%
%\renewcommand*{\appendixattoc}{% Trennanweisung im Inhaltsverzeichnis
%\setcounter{tocdepth}{\@alltocdepth}% tocdepth wiederherstellen
%}%
%\tableofcontents% Verzeichnis ausgeben
%\setcounter{tocdepth}{\@alltocdepth}% tocdepth wiederherstellen
%\endgroup
%}
%\newcommand*{\appendixattoc}
%\g@addto@macro\appendix{% \appendix erweitern
%\if@openright\cleardoublepage\else\clearpage\fi% Neue Seite
%\addcontentsline{toc}{chapter}{\appendixname}% Eintrag ins Hauptverzeichnis
%\addtocontents{toc}{\protect\appendixattoc}% Trennanweisung in die toc-Datei
%}
%\makeatother

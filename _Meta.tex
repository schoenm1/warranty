% Informationen ------------------------------------------------------------
% 	Definition von globalen Parametern, die im gesamten Dokument verwendet
% 	werden k�nnen (z.B auf dem Deckblatt etc.).
% --------------------------------------------------------------------------
\newcommand{\titel}{Entwickeln von Anwendungen f�r Hand Held}
\newcommand{\untertitel}{...}
\newcommand{\art}{Seminar Arbeit}
\newcommand{\fachgebiet}{Wirtschaftsinformatik}
\newcommand{\autor}{Andreas Gr�nenfelder}
\newcommand{\autorTwo}{Micha Sch�nenberger}
\newcommand{\studienbereich}{Wirtschaftsinformatik}
\newcommand{\matrikelnr}{12 34 56}
\newcommand{\erstgutachter}{Christian Vils}
%\newcommand{\zweitgutachter}{Dipl.-Inf. Lukas Podolski}
\newcommand{\jahr}{2012}

% Eigene Befehle und typographische Auszeichnungen f�r diese
\newcommand{\todo}[1]{\textbf{\textsc{\textcolor{red}{(TODO: #1)}}}}

\newcommand{\AutorZ}[1]{\textsc{#1}}
\newcommand{\Autor}[1]{\AutorZ{\citeauthor{#1}}}

\newcommand{\NeuerBegriff}[1]{\textbf{#1}}

\newcommand{\Fachbegriff}[1]{\textit{#1}}
\newcommand{\Prozess}[1]{\textit{#1}}
\newcommand{\Webservice}[1]{\textit{#1}}

\newcommand{\Eingabe}[1]{\texttt{#1}}
\newcommand{\Code}[1]{\texttt{#1}}
\newcommand{\Datei}[1]{\texttt{#1}}

\newcommand{\Datentyp}[1]{\textsf{#1}}
\newcommand{\XMLElement}[1]{\textsf{#1}}


% Abk�rzungen mit korrektem Leerraum
\newcommand{\vgl}{Vgl.\ }
\newcommand{\ua}{\mbox{u.\,a.\ }}
\newcommand{\zB}{\mbox{z.\,B.\ }}
\newcommand{\bs}{$\backslash$}






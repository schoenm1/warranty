\section{Vom Autor verwendete Software}
\label{sec:Werkzeuge}
Im Folgenden werden die Programme vorgestellt, die der Autor zum Erstellen dieser Arbeit und vor allem zur Entwicklung der Webservices verwendet hat. Soweit es m�glich war, wurden Open-Source-Programme eingesetzt.

\begin{itemize}
\item \textbf{Microsoft Visio} \\
Die EPKs der BAP wurden mit Microsoft Visio erstellt. Der Autor hat zwar verschiedene Open-Source-Programme\footnote{Dia, OpenOffice Draw und die EPC Tools.} ausprobiert, mit denen EPKs erstellt werden k�nnten, die grafischen Ergebnisse waren aber nicht zufriedenstellend. Die Symbole von Visio sehen den "`originalen"' ARIS-Symbolen am �hnlichsten und k�nnen dar�ber hinaus mit zus�tzlichen Informationen wie Dauer und Kosten versehen werden.
\item \textbf{PSPad} \\
F�r die Bearbeitung von verschiedenen (Text-)Dateien wurde der Texteditor PSPad verwendet. Mit diesem konnten \zB auch die regul�ren Ausdr�cke f�r die XML-Schemas entwickelt werden. \\
Website: \url{http://www.pspad.com/}  
\item \textbf{Eclipse} \\
Sowohl der ActiveBPEL Designer als auch die EntireX Workbench sind Plugins f�r die IDE Eclipse. Auch zur Java- und PHP-Entwicklung wurde dieses Werkzeug verwendet. \\
Website: \url{http://www.eclipse.org/}  
\item \textbf{XML Copy Editor} \\
F�r die Entwicklung der XML-Schemas und die Bearbeitung von XML-Dateien wurde der XML Copy Editor eingesetzt. Mit diesem k�nnen \ua XML-Dateien auf Wohlgeformtheit gepr�ft und gegen ihr Schema validiert werden. \\
Website: \url{http://xml-copy-editor.sourceforge.net/}
\item \textbf{soapUI} \\
Mit soapUI k�nnen Webservices getestet werden, ohne einen Client zu programmieren. Die SOAP-Anfragen werden automatisch anhand der WSDL generiert und die Antworten k�nnen gegen die WSDL-Datei validiert werden. \\
Website: \url{http://www.soapui.org/}
\item \textbf{Ethereal} \\
Die Netzwerkkommunikation beim Aufrufen der Webservices wurde mit Ethereal, einem umfangreichen Werkzeug zur Analyse des Netzwerkverkehrs, mitgeschnitten. \\
Website: \url{http://www.ethereal.com/}
\item \textbf{\LaTeX} \\
Diese Arbeit wurde mit {\LaTeX} geschrieben. Als Distribution wurde MiKTeX verwendet und als Editor der LaTeX Editor. \\
Websites: \url{http://miktex.org/}, \url{http://www.latexeditor.org/}
\end{itemize}

\chapter{Einleitung}
\label{cha:Einleitung}
TESTTESTTEST\\
Die \NeuerBegriff{Service-orientierte Architektur} (SOA) ist seit einiger Zeit \textit{das} Schlagwort im Bereich der Informationstechnologie. So haben \zB Deutschlands gr��te Softwarehersteller SAP und die Software AG ihre Unternehmensstrategie komplett auf die SOA auscultation. \Autor{SAP2007} bietet mit \Fachbegriff{Netweaver} seine marktf�hrende ERP-Software auf Basis von SOA an,\footnote{\vgl\citep[S.~127]{SAP2007}} und die \Autor{Software2007b}, die sich selbst als "`The XML Company"' bezeichnet, erweiterte k�rzlich noch einmal ihr bereits durchg�ngig an der SOA orientiertes Produktportfolio durch den Kauf des amerikanischen Unternehmens webMethods um L�sungen zur Unterst�tzung von Gesch�ftsprozessen.\footnote{\vgl\citep{Software2007b}} In einem Atemzug mit der SOA werden h�ufig Webservices genannt, da sie durch ihre hohe Plattformunabh�ngigkeit und den Einsatz von Internettechnologie oftmals als Referenzimplementierung f�r die Services in einer SOA angef�hrt werden. Doch welche Vorteile bietet der Einsatz von Webservices in Unternehmen? K�nnen mit ihnen tats�chlich flexiblere Softwaresysteme entwickelt werden? Und wie einfach ist die Implementierung von Webservices auf unterschiedlichen Plattformen? Diesen Fragen wird sich der Autor in der vorliegenden Arbeit widmen.

\section{Motivation/Ausgangslage}
\label{sec:Motivation/Ausgangslage}
UNSERE MOTIVATION

\section{Ziel der Arbeit}
\label{sec:ZielDerArbeit}
UNSER ZIEL DER ARBEIT

\section{Aufbau der Arbeit}
\label{sec:AufbauDerArbeit}
AUFBAU DER ARBEIT...
\section{Voraussetzungen zum Verst�ndnis der Arbeit}
IST DIESER TEIL NOTWENDIG?!?!

\section{Aufgabenstellung}
\label{sec:Aufgabenstellung}
UNSERE AUFGABENSTELLUNG...

\section{Erwartetes Resultat}
\label{sec:ErwartetesResultat}
UNSERE AUFGABENSTELLUNG...

\section{Geplante Termine}
\label{sec:GeplanteTermine}
\begin{tabular}[t]{|l|l|} \hline
 \rowcolor{darkgrey} &\\
\rowcolor{darkgrey}
\multirow{-2}{1.7cm}{\textbf{Datum}} &
\multirow{-2}{10cm}{\textbf{Bezeichnung}} \\
3. Oktober & einschreiben des Projektes im EBS \\ \hline 
5. Dezember & gegeben \\  \hline 
12. Dezember & Arbeitstreffen \\  \hline 
9. Januar & Abgabe der schriftlichen Dokumentation \\  \hline 
16. Januar & Pr�sentation \\  \hline 
\end{tabular}\\



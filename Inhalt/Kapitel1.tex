\chapter{Einf�hrung in Webservices}
\label{cha:Einf�hrungInWebservices}
Wie bereits in Kapitel \ref{cha:Einleitung} erw�hnt, ist zur Unterst�tzung von Gesch�ftsprozessen der Einsatz von Informationstechnologie notwendig. Der Autor verfolgt mit dieser Arbeit das Ziel, einen Gesch�ftsprozess mit Hilfe von Webservices zu optimieren. Hierzu wird er in diesem Kapitel eine Einf�hrung in das Thema Webservices und die damit in Zusammenhang stehenden Technologien geben, und auch auf m�gliche Einsatzbereiche von Webservices im Rahmen der Gesch�ftsprozessoptimierung eingehen.

\section{Definitionen}
\label{sec:DefinitionenWebservices}
Die Service-orientierte Architektur ist ein Ansatz der Softwareentwicklung, der sich stark am Konzept der Gesch�ftsprozesse orientiert und mit Hilfe von Webservices implementiert werden kann. In den beiden folgenden Kapiteln werden beide Begriffe eingehend erl�utert, worauf in Kapitel \ref{cha:Fazit} die f�r die Umsetzung von Webservices ben�tigten Technologien vorgestellt werden.

\subsection{Service-orientierte Architektur}
\label{sec:DefinitionSOA}
\Autor{OASIS2007}\footnote{Die \NeuerBegriff{Organization for the Advancement of Structured Information Standards} ist nach \citep{OASIS2007} ein internationales Konsortium aus �ber 600 Organisationen, das sich der Entwicklung von E-Business-Standards verschrieben hat. Mitglieder sind \zB IBM, SAP und Sun.} definiert den Begriff \NeuerBegriff{Service-orientierte Architektur} (SOA) wie folgt:
\begin{quote}
"`\textbf{Service Oriented Architecture} [\ldots] is a paradigm for organizing and utilizing distributed \textbf{capabilities} that may be under the control of different ownership domains."'\footnote{\citep[S.~8]{OASIS2006a}}
\end{quote}
Diese bewusst allgemein gehaltene Definition stammt aus dem Referenzmodell der SOA aus dem Jahr 2006. Dieses Modell wurde mit dem Ziel entwickelt, ein einheitliches Verst�ndnis des Begriffs SOA und des verwendeten Vokabulars zu schaffen, und sollte die zahlreichen bis dato vorhandenen, teils widerspr�chlichen Definitionen abl�sen.\footnote{\vgl\citep[S.~4]{OASIS2006a}} Dabei wird zun�chst noch kein Bezug zur Informationstechnologie hergestellt, sondern allgemein von F�higkeiten gesprochen, die Personen, Unternehmen, aber eben auch Computer besitzen und evtl. Anderen anbieten, um Probleme zu l�sen. Als Beispiel wird ein Energieversorger angef�hrt, der Haushalten seine F�higkeit Strom zu erzeugen anbietet.\footnote{\vgl\citep[S.~8f.]{OASIS2006a}}

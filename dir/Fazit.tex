\chapter{Fazit}
\label{sec:Fazit}
R�ckblickend k�nnen wir sagen, dass die App- Programmierung f�r Android basierte Devices ein sehr spannendes Thema ist. Nach einem gewissen Initialaufwand gelingt es auch Einsteigern schnell, eine kleine App mit ein paar wenigen Funktionen zu programmieren. Schnell konnten wir feststellen, dass mit ein bisschen �bung die M�glichkeiten er App- Programmierung auf Android fast unbegrenzt sind. Nach dem Einarbeiten steigt die Lernkurve rasant an bevor sie dann mit der Zeit abzuflachen beginnt.

Dank diversen Webseiten, Blogs und PDFs die im Internet frei verf�gbar sind, kann der eigene Code mit anderem verglichen und allenfalls optimiert werden. Die verf�gbaren Beispiele rufen nicht selten gute Ideen f�r die eigene App hervor. So kam es beispielsweise, dass wir die  urspr�nglichen, langweiligen Datumsfelder durch optisch ansprecherende DatePickers ersetzen konnten. 

Da bei dieser Seminararbeit vor allem auch die Dokumentation im Vordergrund stand, war dies eine optimale Gelegenheit sich auf die kommende Semester- bzw. Bachelorarbeit vorzubereiten. Diese nutzten wir, indem wir uns zum Ziel setzten, die gesamte Arbeit in Latex und gem�ss Best- Practice f�r technische Dokumentation zu schreiben. 


\textcolor{red}{!!!! HIER KOMMT NOCH TEXT !!!!}

\section{Punkt 1}

\textcolor{red}{!!!! HIER KOMMT NOCH TEXT !!!!}

\subsection{Punkt 1.1}

\textcolor{red}{!!!! HIER KOMMT NOCH TEXT !!!!}

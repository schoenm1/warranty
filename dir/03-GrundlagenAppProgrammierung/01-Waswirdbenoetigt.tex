\section{Was wird f�r die App Programmierung ben�tigt}

\textbf{Es gibt mehrere M�glichkeiten, ein Android App zu programmieren. Neben Software, die es erlaubt, offline zu programmieren, gibt es auch online Tools, bei welchen man keine bis sehr wenig Programmierkenntnisse ben�tigt. \\
Unsere Vorkenntnisse in der Programmiersprache Java basieren  auf den beiden Module Programmieren 1 und Programmieren 2 des Grundstudiums.
\\
-grosse Palette (einige Beispiele)
\\
Da wir bereits im Grundstudium die opensource Programmiersoftware Eclipse nutzten, beschr�nken wir uns hier auch auf diese Software.\\
}
\subsection{Eclipse (IDE)}

\textbf{
Eclipse} \footnote{offizielle Website: \url{http://de.wikipedia.org/wiki/Eclipse_(IDE)}} \cite{wiki_eclipse} (vom englischen eclipse = Sonnenfinsternis hergeleitet) ist ein open-source Programmierwerkzeug. Zu Beginn wurde Eclipse als eine Entwiklungsumgebung f�r Java entwickelt. Im Laufe der Zeit hat sich Eclipse weiterentwickelt und durch die M�glichkeit der Skalierbarkeit wurde vom Java-Programmiertool ein Werkzeug, welches f�r viele Entwicklungsaufgaben eingesetzt werden kann. Die grosse Community und der modulare Aufbau, welche die Weiterentwicklung vom Modulen und Plugins  immer vorantreiben, haben aus diesem Tool ein m�chtiges Tool gemacht, welches sich f�r den Entwickler individuell zuschneiden l�sst.
Es gibt f�r Eclipse mittlerweile open-souce sowie auch kommerzielle Erweiterungen.
Eclipse selbst basiert auf Java-Technik, seit Version 3.0 auf einem sogenannten OSGi-Framework namens Equinox.\\
Speziell f�r die Entwicklung von Android Applikationen
existiert das ADT (Android Development Tools) Plug-in. Dieses Plug-in erweitert den
Funktionsumfang von Eclipse und erm�glicht somit ein einfaches Entwickeln von Android
Projekten.
Eclipse wurde als Grundwerkzeug f�r die App-Programmierung benutzt. In den ersten beiden Studienjahren haben wir mit Eclipse Java-Applikationen entwickelt.\\ \\

\begin{figure}[h]
\centering
\includegraphics[height=5cm]{eclipse_logo.png} \\
\caption{Logo Eclipse}
\label{Logo_Eclipse}
\end{figure}


\subsection{Android SDK Plugin  for Eclipse}

Android Software Development Kit (SDK) \footnote{\url{http://developer.android.com/sdk/index.html}} \cite{andoid_sdk_eclipse} ist ein Plugin f�r Eclipse IDE welches als m�chtige, integrierte Entwicklungsumgebung konzipiert wurde, um Android Applikationen zu entwickeln.\\
Will man beginnen, Android Apps zu programmieren, kommt man um das Android SDK nicht herum.\\
Die Android SDK gibt es f�r Windows, Mac OS X sowie Linux Plattformen.\\
Um Android SDK nutzen zu k�nnen, ist die Java SDK (Software Development Kit) unabdingbar. Diese ist je nach Betriebssystem bereits vorinstalliert oder kann nachtr�glich heruntergeladen und installierte werden.

offizielle Website:  \url{http://developer.android.com}\\
Anleitung Installation Plugin: \url{http://developer.android.com/sdk/installing}

\subsection{Testing}
Da die Zeit f�r dieses Projekt nicht ausreicht f�r ein ausgiebiges Testen mit JUnit- und JMock-Klassen, haben wir uns auf ein Testing beschr�nkt auf das Live-Testing.
Zur Auswahl standen:
\begin{itemize}
\item Galaxy Nexus, Version 4.0.1
\item Samsung \textcolor{red}{GR�ENI}
\item virtuelle Maschine, welche in Eclipse integriert ist und sich wahlweise die Android Version, aber auch Telefontyp �ndern l�sst
\end{itemize}

\newpage
\section{App-Vermarkung}
\index{App-Vermarktung}
Online gibt es hunderte von Abhandlungen �ber die Vermarktungs-Strategie von Android Applikationen.\\
Es gibt Anleitungen wie man seine App unter die 100 Besten im Play Store \index{Play Store} bringen kann, wie man den Preis festlegen soll, so dass der maximale Profit erwirtschaftet werden kann, wie man sich einen Namen macht als Entwickler... Da sind keine Grenzen gesetzt.\\ \\
zwei Beispiele:\\
\url{http://www.androidpit.de/Strategische-Herangehensweise-bei-der-App-Entwicklung-Die-Idee-Teil-1}\\
\url{http://theappencypress.com/2010/08/04/everything-you-need-to-know-about-being-an-android-app-seller/}


\subsection{Android Market - Google Play Store}
 \index{Play Store|(}
\index{Android Market|see{Play Store}} \index{Google Play Store|see{Play Store}}

Die wohl bekannteste Variante ist der Android Market. Seit dem 06. M�rz 2012 heisst dieser offiziell Google Play \cite{google_playstore} Store.
\footnote{offizielle Website: \url{https://play.google.com}} 
Gem�ss Angaben von Wikipedia\cite{google_playstore_wiki} waren Ende Januar 2012 �ber 360'000 Anwendungen verf�gbar, welche insgesamt �ber 10 Milliarden mal heruntergeladen wurden. Zirka 15\% der Anwendungen sind Spiele. Der Umsatz betr�gt mehr als 5 Millionen US-Dollar pro Monat.\\

Um eine Anwendung auf dem Play Store anbieten zu k�nnen, muss man sich als Entwickler auf \url{https://play.google.com/apps/publish/signup} registrieren und einmalig 25 US-Dollar bezahlen. Nach der Registrierung stehen einem T�r und Tor offen. Eigene Applikationen k�nnen nun kostenlos oder zu einem selbst ernannten Preis vermarkten. Momentan sind etwa 65\% der Applikationen kostenlos verf�gbar.\\
Google verlangt, genauso wie Apple und Microsoft, eine Transaktionsgeb�hr von 30\% des Verkaufswert.\\
\index{Play Store|)}
\newpage
Es gibt jedoch auch Alternativen\cite{google_playstore_wiki} zu Googles Play Store:
\begin{itemize}
\item \textbf{F-Droid}  \index{F-Droid}
Ein Appstore, der als non-profit Projekt von einer Community freiwilliger Unterst�tzer betrieben wird und �ber den ausschlie�lich kostenlose, freie Software-Apps (Open Source) bereitgestellt werden.\\
\url{http://f-droid.org/}

\item \textbf{SlideME}  \index{SlideME}
SlidME bietet eine Plattform f�r Entwickler und Benutzer von Android. Entwickler k�nnen ihre Applikation kostenlos oder auch kostenpflichtig bei SlideME ver�ffentlichen. Der SlideME Market umfasst ca. 2000 Applikationen.\\
\url{http://www.slideme.org}

\item \textbf{AndroidPIT} \index{AndroidPIT}
AndroidPIT betreibt einen eigenen Store und bietet auch anderen Unternehmen diesen Store f�r ihre Android-Ger�te an. Hierzu geh�ren Unternehmen wie 1\&1, Telefunken, Pearl, Point of View und Interpad.\\
\url{http://www.androidpit.de/}

\item ... und noch viele mehr ...

\end{itemize}

\subsection{public link} \index{public link}

Will man seine Applikation nicht in einem online-Store wie Google Play Store anbieten, kann man sie auch direkt verkaufen oder kostenlos anbieten.\\
Hier liegt ein Vorteil von Google gegen�ber anderen Anbietern, wie zum Beispiel Apple. Auf jedem Ger�t, auf welchem die Android Plattform als Betriebssystem l�uft, kann man die Funktion ein- und ausschalten, welche es erlaubt, auch Applikationen von Fremdanbietern (also nicht Play Store von Google) zu installieren. Daf�r ist kein Jailbreak oder dergleichen notwendig.\\
Jeder, der diese Funktion nun eingeschaltet hat, kann die Applikation von jeglichem Webserver auf der Welt herunterladen.\\ \\
\textbf{Sicherheitshinweis:} Hier sollte man unbedingt beachten, dass man auch sch�dliche Software installieren kann, wenn man erlaubt, aus nicht Google-Seiten Applikationen zu laden!


\subsection{Mail, Stick...} \index{Mail} \index{Stick}

Als dritte Variante bietet sich die direkte Vervielf�ltigung an. Diese kann per Mail, USB-Stick oder dergleichen erfolgen. Jede Android Applikation endet mit \textbf{.apk}. Ist das Smartphone am Computer angeschlossen (Windows, Mac OS X, Linux), kann die Applikation ohne Probleme installierte werden.


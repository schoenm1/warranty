\subsection{Manifest.xml} \index{Manifest.xml}
\subsubsection{Permissions} \index{Permissions}
Die Warranty App ben�tigt wenige zus�tzliche Rechte. Um Bilder speichern zu k�nnen ist jedoch ein Speicherplatz erforderlich. Da die meisten Android basierten Smartphones einen sehr kleinen internen Speicher haben, bietet sich der externe Speicher bestens an. Je nach Modell ist dies entweder eine zus�tzlichen MicroSD [Glossar] Karte oder einfach eine zus�tzliche logische Partition auf dem internen Speicher. Der Vorteil dieses externen Speichers ist, dass man ihn ohne weiteres auf dem Computer einh�ngen und auf die Dateien zugreifen kann.

Um auf diesen externen Speicher zuzugreifen, wird folgende Zeile im Manifest.xml ben�tigt.
\lstset{language=Java, numbers=left}


\begin{lstlisting}[label=lst:latex.listing,captionpos=b, caption={Android permission f�r Zugriff auf externen Storage, Manifest.xml}]
<uses-permission android:name="android.permission.WRITE_EXTERNAL_STORAGE" />
\end{lstlisting}

Nebst dem Zugriff auf den Speicher, bedient sich Warranty den Kamerafunktionalit�ten. Da diese ebenfalls explizit erlaubt werden m�ssen, wird das entsprechende Recht im Manifest.xml hinterlegt.
\lstset{language=xml, numbers=left}
\begin{lstlisting}[label=lst:latex.listing,captionpos=b, caption={Android permission f�r Kamera, Manifest.xml}]
<uses-permission android:name="android.permission.CAMERA" />
\end{lstlisting}

\newpage
\subsubsection{Application}
Zus�tzlich werden in der Manifest.xml s�mtliche Activities hinterlegt, die von der App ausgef�hrt werden m�ssen. Das diese Liste vollst�ndig und korrekt ist, ist f�r die App �berlebensnotwendig. Wird im Code eine App Activity aufgerufen, die in dieser Sektion nicht aufgef�hrt ist, st�rzt die gesammte App ab.

\lstset{language=xml, numbers=left}
\begin{lstlisting}[label=lst:latex.listing,captionpos=b, caption={Deklaration der Activities, Manifest.xml}]
   <application
        android:icon="@drawable/icon"
        android:label="@string/app_name" >
        <activity
            android:name="ch.zhaw.warranty.CardListActivity"
            android:label="@string/app_name" >
            ...
        </activity>
        <activity android:name="ch.zhaw.warranty.CardActivity" />
        <activity android:name="ch.zhaw.warranty.photo.PhotoActivity" />
        <activity android:name="ch.zhaw.warranty.photo.PhotoDisplayActivity"/>
    </application>
\end{lstlisting}

Nebst den zu erlaubenden Activities wird zus�tzlich die zu startende Activity, in Programmierjargon ``Main'' definiert.
\lstset{language=xml, numbers=left}
\begin{lstlisting}[label=lst:latex.listing,captionpos=b, caption={Deklaration der Main- Methode, Manifest.xml}]
        <activity
            android:name="ch.zhaw.warranty.CardListActivity"
            android:label="@string/app_name" >
            <intent-filter>
                <action android:name="android.intent.action.MAIN" />	
                <category android:name="android.intent.category.LAUNCHER" />
            </intent-filter>
\end{lstlisting}
\subsection{Multi Language Support} \index{Multi Language Support}
Um Warranty Multi Language Supportive zu machen, wurden s�mtliche Buttons und Textfelder dynamisch, mit Referenz auf die Datei strings.xml benannt. 
Eine explizite Selektion der Sprache wurde nicht programmiert, da das App grunds�tzlich immer in der Sprache des Smartphones, also der Locale entsprechen soll. 

In den XML- Dateien werden die Strings aus der \textit{strings.xml} mittels \textit{@string\/} referenziert. 
\lstset{language=XML, numbers=left}
\begin{lstlisting}[label=lst:latex.listing,captionpos=b, caption={Definition Textfeld "description", activity\_card.xml}]
<EditText
	android:id="@+id/card.TBdesc"
	...
	android:hint="@string/description" />
\end{lstlisting}

Anschliessend k�nnen werden diese in der \textit{strings.xml} in dem dazugeh�rigen Sprachordner aufgelistet.
\lstset{language=XML, numbers=left}
\begin{lstlisting}[label=lst:latex.listing,captionpos=b, caption={String Definition in Deutsch, values-de/string.xml}]
<resources>
	<string name="description">Beschreibung</string>
	...
</resources>
\end{lstlisting}

Wichtig ist hierbei vor allem, dass eine Standard- Locale existiert. Diese ist immer im Ordner ``values'' zu finden.
\lstset{language=XML, numbers=left}
\begin{lstlisting}[label=lst:latex.listing,captionpos=b, caption={String Definition in Englisch, values/string.xml}]
<resources>
	<string name="description">Description</string>
	...
</resources>
\end{lstlisting}

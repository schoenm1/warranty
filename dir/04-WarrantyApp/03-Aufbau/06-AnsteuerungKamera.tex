\newpage

\subsection{Ansteuerung der Kamera} \index{Kamera|(}
Die Kamera stellt in Warranty eines der zentralen Teile dar, da die Hauptfunktion der App auf dem Speichern von Fotos basiert. Dementsprechend ist es wichtig, dass der Code, der diese Funktionalit�t bereitstellt �bersichtlich und nicht aufgebl�ht ist und stabil l�uft.

\subsubsection{Externen Storage einbinden}
Damit die aufgenommen Bilder in zuk�nfitgen Releasen einfach exportiert werden k�nnen, werden diese in einem eigenen Directory auf dem externen Storage gespeichert. Dies hat zus�tzlich den Vorteil, dass der User beim anschliessen des Smartphones an den Computer dieses als externes Storage- Device mounten und somit einfach auf die aufgenommen Bilder zugreifen kann.

Zu Beginn wird der Zielspeicherort f�r die Bilder definiert und anschliessend gepr�ft ob
\begin{enumerate}
\item die SDCard gemountet ist
\item das Speicherverzeichnis existiert
\item das Speicherverzeichnis, falls inexistent, erstellt werden kann
\end{enumerate}
Mit dieser Prozedur kann sichergestellt werden, dass die Kamera- Applikation Bilder schreiben kann.

\lstset{language=Java, numbers=left}
\begin{lstlisting}[label=lst:tbl_attr_names, captionpos=b, caption={Zugriff auf SDCard, PhotoActivity.java}]
private File getPhotoDir() {
	File storageDir = new File(Environment.getExternalStorageDirectory() + "/Warranty");
	if (Environment.MEDIA_MOUNTED.equals(Environment.getExternalStorageState())) {
		if (storageDir != null) {
			if (!storageDir.mkdirs()) {
				if (!storageDir.exists()) {
					Log.d("getPhotoDir()", "failed to create directory");
					return null;
				}
				...
			...
		return storageDir;
\end{lstlisting}

\subsubsection{Bilddatei erstellen}
Da der von Android bereitgestellten Kamera- Applikation anschliessend ein leeres File �bergeben werden muss, ist der n�chste Schritt, ein leeres File zu erstellen. \\
Um die Eindeutigkeit des Files sicherzustellen, bedient sich Warranty den beiden Java- eigenen Klassen \textit{Date} und  \textit{SimpleDateFormat}. Mit Hilfe der Klasse \textit{SimpleDateFormat} wird der Dateinamen wie folgt aufgebaut: \textbf{yyyyMMdd\_HHmmss}. Die von Warranty verwendeten Buchstaben haben gem�ss Javadoc der Klasse  SimpleDateFormat\footnote{\url{http://docs.oracle.com/javase/6/docs/api/java/text/SimpleDateFormat.html}} folgende Bedeutung:\\

\begin{tabbing}
\hspace{50px}\=\kill
y \>Year \\
M \>Month \\
d \>Day \\
H \>Hour \\
m \>Minute \\
s \>Second \\
\end{tabbing}
\textbf{Beispiel:}\\
Der 12. Oktober 2012 zur Zeit 15:27:42 Uhr wird als 20121012\_152742 dargestellt.

\index{Kamera|)}

\newpage

\subsection{Ansteuerung der Kamera} \index{Kamera|(}

Die Kamera stellt in Warranty eines der zentralen Teile dar, da die Hauptfunktion der App auf dem Speichern von Fotos basiert. Dementsprechend ist es wichtig, dass der Code, der diese Funktionalit�t bereitstellt �bersichtlich und nicht aufgebl�ht ist und stabil l�uft.

\subsubsection{Externen Storage einbinden}
Damit die aufgenommen Bilder in zuk�nfitgen Releasen einfach exportiert werden k�nnen, werden diese in einem eigenen Directory auf dem externen Storage gespeichert. Dies hat zus�tzlich den Vorteil, dass der User beim anschliessen des Smartphones an den Computer dieses als externes Storage- Device mounten und somit einfach auf die aufgenommen Bilder zugreifen kann.

Zu Beginn wird der Zielspeicherort f�r die Bilder definiert und anschliessend gepr�ft ob
\begin{enumerate}
\item die SDCard gemountet ist
\item das Speicherverzeichnis existiert
\item das Speicherverzeichnis, falls inexistent, erstellt werden kann
\end{enumerate}
Mit dieser Prozedur kann sichergestellt werden, dass die Kamera- Applikation Bilder schreiben kann.

\lstset{language=Java, numbers=left}
\begin{lstlisting}[label=lst:tbl_attr_names, captionpos=b, caption={Zugriff auf SDCard, PhotoActivity.java}]
private File getPhotoDir() {
	File storageDir = new File(Environment.getExternalStorageDirectory() + "/Warranty");
	if (Environment.MEDIA_MOUNTED.equals(Environment.getExternalStorageState())) {
		if (storageDir != null) {
			if (!storageDir.mkdirs()) {
				if (!storageDir.exists()) {
					Log.d("getPhotoDir()", "failed to create directory");
					return null;
				}
				...
			...
		return storageDir;
\end{lstlisting}

\subsubsection{Bilddatei erstellen}
Da der von Android bereitgestellten Kamera- Applikation anschliessend ein leeres File �bergeben werden muss, ist der n�chste Schritt, ein leeres File zu erstellen. \\
Um die Eindeutigkeit des Files sicherzustellen, bedient sich Warranty den beiden Java- eigenen Klassen \textit{Date} und  \textit{SimpleDateFormat}. Mit Hilfe der Klasse \textit{SimpleDateFormat} wird der Dateinamen wie folgt aufgebaut: \textbf{yyyyMMdd\_HHmmss}. Die von Warranty verwendeten Buchstaben haben gem�ss Javadoc der Klasse  SimpleDateFormat\footnote{\url{http://docs.oracle.com/javase/6/docs/api/java/text/SimpleDateFormat.html}} folgende Bedeutung:\\



\begin{table}[h!]
\centering
\caption{geplante Termine}
\label{tab:Zeitvariabeln}
\begin{tabular}[t]{|l|l|l|} \hline
\cellcolor{darkgrey} &  \cellcolor{darkgrey} &  \cellcolor{darkgrey} \\ 
\cellcolor{darkgrey} \multirow{-2}{2cm}{\textbf{Buchstabe}} &
\cellcolor{darkgrey} \multirow{-2}{4cm}{\textbf{Beschreibung EN}} & 
\cellcolor{darkgrey} \multirow{-2}{4cm}{\textbf{Beschreibung DE}}  \\  \cline{1-3}
y & Year & Jahr \\ \cline{1-3}
M & Month & Monat \\ \cline{1-3}
d & Day & Tag \\   \cline{1-3}
H & Hour & Stunde \\  \cline{1-3}
m & Minute & Minute \\  \cline{1-3}
s & Second & Sekunde \\  \cline{1-3}
\end{tabular}
\end{table}


%\begin{tabbing}
%\hspace{50px}\=\kill
%y \>Year \\
%M \>Month \\
%d \>Day \\
%H \>Hour \\
%m \>Minute \\
%s \>Second \\
%\end{tabbing}
\textbf{Beispiel:}\\
Der \textit{12. Oktober 2012} zur Zeit \textit{15:27:42 Uhr} wird als \textit{20121012\_152742} dargestellt.

In Java kann das erstellen eines leeren Files inklusive Zeitstempel im Filenamen mit ein paar wenigen Zeilen gemacht werden.
\lstset{language=Java, numbers=left}
\begin{lstlisting}[label=lst:tbl_attr_names, captionpos=b, caption={leeres File erstellen PhotoActivity.java}]
private static final String JPEG_FILE_PREFIX = "IMG_";
private static final String JPEG_FILE_SUFFIX = ".jpg";
...
private File createImageFile() throws IOException {
	String timeStamp = new SimpleDateFormat("yyyyMMdd_HHmmss").format(new Date());
	String imageFileName = JPEG_FILE_PREFIX + timeStamp;
	File albumF = getPhotoDir();
	File imageF = File.createTempFile(imageFileName, JPEG_FILE_SUFFIX, albumF);
}
\end{lstlisting}

\subsubsection{Foto aufnehmen}
Mit dem �berpr�fen der Verf�gbarkeit der SDCard sowie dem erstellen einer leeren Bilddatei sind die Bedingungen f�r das Aufnehmen eines Fotos erf�llt. Im n�chsten Schritt kann die von Android zur Verf�gung gestellte Kamera- Applikation mit dem leeren File als Parameter gestartet werden. 

\lstset{language=Java, numbers=left}
\begin{lstlisting}[label=lst:tbl_attr_names, captionpos=b, caption={}]
Intent takePictureIntent = new Intent(MediaStore.ACTION_IMAGE_CAPTURE);
switch (actionCode) {
case ACTION_TAKE_PHOTO_B:
	File f;
	try {
		f = createImageFile();
		mCurrentPhotoPath = f.getAbsolutePath();
		takePictureIntent.putExtra(MediaStore.EXTRA_OUTPUT,	Uri.fromFile(f));
		...
	...
}
startActivityForResult(takePictureIntent, actionCode);
\end{lstlisting}
Die Kontrolle gelangt erst nach erfolgreichem speichern des Bildes durch die Kamera- Applikation wieder zur�ck zu Warranty. Warranty �ffnet sogleich die Activity \textit{CardActivity} mit dem Pfad der gespeicherten Datei und dem Status ``new'' als Parameter.

\lstset{language=Java, numbers=left}
\begin{lstlisting}[label=lst:tbl_attr_names, captionpos=b, caption={}]
protected void onActivityResult(int requestCode, int resultCode, Intent data) {
	switch (requestCode) {
	case ACTION_TAKE_PHOTO_B: {
		if (resultCode == RESULT_OK) {
			Bundle imagePath = new Bundle();
			imagePath.putString("path", mCurrentPhotoPath);
			Intent intent = new Intent(PhotoActivity.this, ch.zhaw.warranty.CardActivity.class);
			intent.putExtra("path",mCurrentPhotoPath);
			intent.putExtra("status","new");
			startActivity(intent);
	...
}
\end{lstlisting}
Von \textit{CardActivity} werden anschliessend die \textit{EditText} Felder geparst und die in Kapitel \refTC{sec:04-insert_and_update} vorgestellte \textit{insertWarrantyCard}- Methode �bergeben.
\index{Kamera|)}
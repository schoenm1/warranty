\newpage
\subsection{Datenbank} \index{Datenbank} \index{DBMS} \index{Datenbank|(}
Auch wenn wir auf unserem Smartphone eine Datenbank ben�tigen, so scheint die Idee, ein vollumf�ngliches \abkDef{DBMS}{\textbf{D}atabase \textbf{M}anagement \textbf{S}ystem}  wie beispielsweise MySQL\footnote{offizielle Website: \url{http://www.mysql.com}}\index{MySQL} oder PostgreSQL\footnote{offizielle Website: \url{http://www.postgresql.org}}\index{PostgreSQL} zu installieren absurd. Zum einen ben�tigen wir Features wie ein Client/Server Model, Partitioning oder ein ausgefeiltes Zugriffsberechtigungssystem nicht, zum anderen steht die dazu ben�tigte Performance auf einem Smartphone schlicht und einfach nicht zur Verf�gung. 
\newline
Um dennoch eine Datenbank auf einem Smartphone verwenden zu k�nnen, bietet sich SQLite\footnote{offizielle Website: \url{http://www.sqlite.org}}\index{SQLite} an. SQLite ist eine Programmbibliothek, die sich direkt in der Applikation einbinden l�sst und somit keinen Server- Prozess ben�tigt, also ressourcensparend ist. Die gesamte Datenbank inklusive aller Tabellen, Indizes und Werten werden in einer einzigen Datei abgelegt, was ein paralleles Schreiben auf die Datenbank unm�glich macht.
\newline
Dank der nativen SQLite Unterst�tzung von Android, f�llt ein aufw�ndiges einbinden einer 3rd Party Library weg.
\\
Um die Datenbankfunktionalit�t bereitzustellen, wurde ein eigenes Java- Package\\ \textit{ch.zhaw.warranty.database} erstellt. Die darin enthaltenen Klassen sind f�r das Erstellen der Datenbank und deren Tabellen (\textit{TBLWarrantyHelper}) beziehungsweise zur Bereitstellung der Datenbankfunktionen f�r den User (\textit{TBLWarrantyConnector}) zust�ndig.


\newpage
\subsubsection{Definition der Datenbank-Tabelle} \index{Datenbanktabelle|(}
Um aus s�mtlichen Java- Klassen auf den Tabellen- sowie die Attributsnamen zugreifen zu k�nnen und um ein statisches Referenzieren in den Java- Klassen zu vermeiden, macht es Sinn, diese als statische Strings zu definieren.
\lstset{language=Java, numbers=left}
\begin{lstlisting}[label=lst:tbl_attr_names, captionpos=b, caption={Deklaration des Tabellen- sowie der Attributsnamen, TBLWarrantyHelper.java}]
public static final String TBL_NAME="warranty";
public static final String CLMN_ID = "_id";
public static final String CLMN_TITLE = "title";
public static final String CLMN_DESC = "description";
public static final String CLMN_IMGPATH = "img_path";
public static final String CLMN_CREATEDAT = "created_at";
public static final String CLMN_VLDTIL = "valid_until";
public static final String CLMN_PRICE = "price";
public static final String CLMN_RESSELLER = "reseller";
\end{lstlisting}

Darauffolgend wird das SQL-Statement zur Erstellung der Tabelle definiert. Die im Listing \refTC{lst:tbl_attr_names} aufgef�hrten Strings k�nnen hier bereits als Referenz benutzt werden.
\lstset{language=Java, numbers=left}
\begin{lstlisting}[label=lst:latex.listing,captionpos=b, caption={Vorbereiten Erstellungsbefehls, TBLWarrantyHelper.java}]
private static final String DB_CREATE="create table " + TBL_NAME + " (" +
	CLMN_ID + " integer primary key autoincrement," + 
	CLMN_TITLE + " text," + 
	CLMN_DESC + " text," + 
	CLMN_IMGPATH + " text," + 
	CLMN_CREATEDAT + " text," +
	CLMN_VLDTIL + " text," +
	CLMN_PRICE + " real," +
	CLMN_RESSELLER + " text);";
\end{lstlisting}

\newpage
Abschliessend wird die \textit{onCreate}-Methode \index{onCreate-Methode} der Klasse \textit{Activity} �berschrieben, so dass bei jedem erstellen  der Activity das Tabellen- Erstelungsstatement aufgerufen wird. Existiert die Tabelle bereits, so hat das Statement keinen Einfluss, es wird von SQLite ignoriert.
\lstset{language=Java, numbers=left}
\begin{lstlisting}[label=lst:latex.listing,captionpos=b, caption={Tabelle erstellen, TBLWarrantyHelper.java}]
@Override
public void onCreate(SQLiteDatabase db) {
	db.execSQL(DB_CREATE);
}
\end{lstlisting}
\index{Datenbanktabelle|)}

\subsubsection{Insert und Update Funktion} \label{sec:04-insert_and_update} \index{Insert} \index{Update}
Die aus Anwendersicht vermutlich wichtigsten Methoden sind die Insert- und die Update- Methoden. Da diese von der Funktionalit�t sehr �hnlich sind, ist aus Sicht des Programmierers sinnvoll, diese zusammen zulegen. Der grundlegende Unterschied ist , dass beim Insert ein neuer Eintrag erstellt, beim Update ein bereits vorhandener Eintrag angepasst wird.

Im Javacode ist die Differenzierung denkbar einfach.
\lstset{language=Java, numbers=left}
\begin{lstlisting}[label=lst:latex.listing,captionpos=b, caption={Insert und Update Funktion, TBLWarrantyConnector.java}]
public void insertWarrantyCard(WarrantyCard card){
	...
	openDB();
		if (card.get_id() == 0) {
			db.insert(TBLWarrantyHelper.TBL_NAME, null, values);
		} else {
			db.update(TBLWarrantyHelper.TBL_NAME, values, TBLWarrantyHelper.CLMN_ID + "=" + card.get_id(), null);
		}
	closeDB();
}
\end{lstlisting}
\newpage
Der Grund f�r diese einfache Unterscheidung liegt in der Auflistung der Quittungen im Homescreen der App. Diese sogenannte Listview \textbf{referenz zum bild} kennt von jeder aufgelisteten Quittung ihre dazugeh�rige ID. M�chte der User eine Quittung bearbeiten, wird in der App intern diese Referenz auf die Quittung weitergegeben. M�chte der User eine neue Quittung hinterlegen, wird das ID Feld nicht gef�llt. Dies f�hrt dazu, dass der Standard Integer- Wert, in Java eine 0\footnote{gem�ss \url{http://docs.oracle.com/javase/tutorial/java/nutsandbolts/datatypes.html}} 

\subsubsection{Delete Funktion} \index{Delete}
Da die Quittungen nach deren Ablauf nicht automatisch gel�scht werden, kann es durchaus sein, dass ein User die Quittungen manuell l�schen m�chte. 

Wie bereits im Kapitel \refTC{sec:04-insert_and_update} erw�hnt, ist der Listview auf dem Homescreen die ID jeder aufgelisteten Quittung bekannt. Somit kann eine Quittung direkt aus der Listview mit dem Methodenaufruf \textit{deleteCard} und der dazugeh�rigen ID, die anschliessend das entsprechende SQL- Statement an die Datenbank kommuniziert, gel�scht werden.

Wie bereits beim Insert und Update haben wir uns auch hier die Tatsache, dass das \textbf{auto increment} von SQLite bei 1 beginnt\footnote{gem�ss \url{http://www.sqlite.org/autoinc.html}} zu nutzen gemacht. Wird 0 als ID �bergeben, hat dies zur Folge, dass ausnahmslos alle gespeicherten Quittungen gel�scht werden.

\lstset{language=Java, numbers=left}
\begin{lstlisting}[label=lst:latex.listing,captionpos=b, caption={Delete Funktion, TBLWarrantyConnector.java}]
public void deleteCard(int cardID) {
	openDB();
	if (cardID == 0) {
		db.delete(TBLWarrantyHelper.TBL_NAME, null, null);		
	} else {
		db.delete(TBLWarrantyHelper.TBL_NAME, TBLWarrantyHelper.CLMN_ID + "=" + cardID, null);
	}
	closeDB();
}
\end{lstlisting}

Eine Funktion zum l�schen aller Quittungen wurde im Men� des Homescreens untergebracht.

 \index{Datenbank|)}

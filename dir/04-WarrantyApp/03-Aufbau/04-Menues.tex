
\subsection{Men�s} \index{Men�s}
Bei Warranty ist nur die CardListActivity, die Hauptactivity mit einem Men� ausgestattet. Die beiden anderen Activities ben�tigen dieses Feature von Android nicht.

Analog zu den Activities werden auch Men�s als XML- Files definiert.

\lstset{language=XML, numbers=left}
\begin{lstlisting}[label=lst:latex.listing,captionpos=b, caption={Definition eines Men�s, activity\_card\_list.xml}]
<item android:id="@+id/cardlist.menu.DeleteAll" android:title="@string/deleteAllCards"></item>
<item android:id="@+id/cardlist.menu.exit" android:title="@string/exit"></item>
\end{lstlisting}

\newpage
In der dazugeh�rigen Activity wird das Men� eingebunden und generiert.
\lstset{language=Java, numbers=left}
\begin{lstlisting}[label=lst:latex.listing,captionpos=b, caption={Erstellen von Men�s, CardListActivity.java}]
@Override
public boolean onCreateOptionsMenu(Menu menu) {
	getMenuInflater().inflate(R.menu.activity_card_list, menu);
	return true;
}
\end{lstlisting}

Besitzt eine Activity kein Men�, so wird dies in der \textit{onCreateOptionsMenu()} einfach nicht generiert.
\begin{lstlisting}[label=lst:latex.listing,captionpos=b, caption={Activity ohne Men�, CardListActivity.java}]
@Override
public boolean onCreateOptionsMenu(Menu menu) {
	return false;
}
\end{lstlisting}

Anschliessend wird in der Activity ausgewertet, welches Men�- Item ausgew�hlt wurde und die entsprechende Aktion ausgef�hrt.
\lstset{language=Java, numbers=left}
\begin{lstlisting}[label=lst:latex.listing,captionpos=b, caption={Men� Listener, CardListActivity.java}]
@Override
public boolean onOptionsItemSelected(MenuItem item) {
	switch (item.getItemId()){
	case R.id.cardlist_menu_exit:
		moveTaskToBack(true);
		break;
	case R.id.cardlist_menu_DeleteAll:
		tblwarranty.deleteCard(0);
		setOrder("title");
	}
	return true;
}
\end{lstlisting}
